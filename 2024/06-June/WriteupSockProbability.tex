\documentclass{article}

\usepackage{amsmath}
\usepackage{amsthm}
\usepackage{amssymb}
\usepackage{amstext}
\usepackage{amsxtra}
\usepackage[margin = 1.0 in]{geometry}
\usepackage{float}
\usepackage{hyperref}

\title{The First Matching Pair of Socks}
\author{Andrew Ford}
\date{\today}

\begin{document}

\maketitle

\section{Link to problem statement}

The link to the problem statement from Zach Wissner-Gross can be found at

\href{https://thefiddler.substack.com/p/can-you-find-a-matching-pair-of-socks}{https://thefiddler.substack.com/p/can-you-find-a-matching-pair-of-socks}.

\section{Original problem: 5 pairs}

We wish to construct a function $f(x)$ such that $f(x)$ is the probability that the first pair of socks is found after drawing exactly $x$ socks. In other words, the first $x-1$ socks drawn are all different, and the next sock drawn after those matches one of the previously drawn socks.

The number of ways to draw $x-1$ socks that all come from different pairs is
${5 \choose {x-1}} \cdot 2^{x-1};$
the first term represents the number of pairs chosen and the second term accounts for having two available socks in each pair. There are a total of ${{10} \choose {x-1}}$ ways to draw $x-1$ socks without regard to pairs, so the probability that the first $x-1$ socks all come from different pairs is
\begin{equation}\label{prob1}
\frac{{5 \choose {x-1}} \cdot 2^{x-1}}{{{10} \choose {x-1}}}.
\end{equation}

After that, there are $10-(x-1) = 11-x$ socks remaining, and $x-1$ of them match a previously drawn sock. Combining this with (\ref{prob1}), we get
\begin{equation}\label{fxexp}
f(x) = \frac{{5 \choose {x-1}} \cdot 2^{x-1}}{{{10} \choose {x-1}}} \cdot \frac{x-1}{11-x}.
\end{equation}
This simplifies to
\begin{equation}\label{fxsimple}
f(x) = \frac{2^{x-1} (x-1) (10-x) (9-x) (8-x) (7-x)}{10 \cdot 9 \cdot 8 \cdot 7 \cdot 6}.
\end{equation}

Obviously, we must draw at least 2 socks to get a pair. Additionally, by the pigeonhole principle, there is guaranteed to be a pair if we draw 6 socks. This means that we only need to compute $f(x)$ for $x \in \{2, 3, 4, 5, 6\}.$ After plugging these values into (\ref{fxsimple}), we get

\begin{table}[H]
\centering
\begin{tabular}{| c | c | c | c | c | c |}
\hline
$x$ & 2 & 3 & 4 & 5 & 6 \\
\hline
$f(x)$ & $7/63$ & $14/63$ & $18/63$ & $16/63$ & $8/63$ \\
\hline
\end{tabular}
\end{table}

Therefore, the probability that the first pair will be found after drawing exactly $x$ socks is maximized when $x = 4.$

\section{Extra credit: N pairs}

To start, we construct a function $g(N, x)$ to represent the probability that, with $N$ pairs of socks in the drawer, we draw the first pair after exactly $x$ socks. Using the same method as in the $N = 5$ case represented by (\ref{fxexp}), we find that
\begin{equation}\label{gnxexp}
g(N,x) = \frac{{N \choose {x-1}} \cdot 2^{x-1}}{{{2N} \choose {x-1}}} \cdot \frac{x-1}{2N + 1 - x}.
\end{equation}
After some simplification, we can choose to rewrite this as
\begin{equation}\label{gnxseparated}
\frac{N!}{(2N)!} \cdot \frac{(2N - x)! \cdot 2^{x-1} \cdot (x-1)}{(N+1-x)!},
\end{equation}
separating out the terms that do not depend on $x$. From here, we define
\begin{align}
h(N,x) &= \frac{g(N,x)}{g(N,x-1)} \\
	&= \frac{\frac{N!}{(2N)!} \cdot \frac{(2N - x)! \cdot 2^{x-1} \cdot (x-1)}{(N+1-x)!}}{\frac{N!}{(2N)!} \cdot \frac{(2N - x + 1)! \cdot 2^{x-2} \cdot (x-2)}{(N+2-x)!}} \\
	&= \frac{2 \cdot (x-1) \cdot (n+2 - x)}{(2N - x + 1) \cdot (x-2)}. \label{hnx}
\end{align}
When $h(N, x)$ is defined in this way, $g(N, x) > g(N, x-1)$ when $h(N, x) > 1$ and $g(N, x) < g(N, x-1)$ when $h(N, x) < 1.$ Since $h(N,x)$ is continuous with respect to $x$ when $x \in (1, 2N-1),$ and it is only reasonable for the probability to be nonzero when $x \in \{2, 3, \ldots, N+1\},$ it is sufficient to solve $h(N,x) = 1$ for $x$ in terms of $N$ to determine candidates for the value of $x$ which maximizes $g(N,x).$

By (\ref{hnx}), we have
\begin{align}
1 &= \frac{2 \cdot (x-1) \cdot (N + 2 - x)}{(2N - x + 1) \cdot (x - 2)} \\
(2N - x + 1) \cdot (x - 2) &= 2 \cdot (x-1) \cdot (N + 2 - x) \\
2Nx - x^2 + x - 4N + 2x - 2 &= 2Nx + 4x - 2x^2 - 2N - 4 + 2x \\
x^2 - 3x - 2N + 2 &= 0 \\
x &= \frac{3 \pm \sqrt{8N + 1}}{2}. \label{quadsol}
\end{align}

Since $N$ is assumed to be ``very large'' in the problem, the solution $\frac{3 - \sqrt{8N + 1}}{2}$ yields a negative value of $x.$ Since only the positive solution makes sense within the constraints of the problem,  it follows that
\begin{equation}
h\left(N, \left\lceil \frac{3 + \sqrt{8N + 1}}{2} \right\rceil \right) < 1 < h\left(N, \left\lfloor \frac{3 + \sqrt{8N + 1}}{2} \right\rfloor \right),
\end{equation}
which means the probability that the first pair is drawn after exactly $x$ socks are pulled from the drawer is maximized when
\begin{equation}
x = \left\lfloor \frac{3 + \sqrt{8N + 1}}{2} \right\rfloor.
\end{equation}

\end{document}